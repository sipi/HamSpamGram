\begin{frame} {Sélection d'attributs}
	\begin{itemize}
		\item Déterminer un sous-ensemble d'attributs pertinents 
		\itemÉtude préalable sur corpus léger : 322 SPAM + 1002 HAM
		\item Tests sur méthodes de recherche + méthode d'évaluation
	\end{itemize}
\end{frame}

\subsection{Etude préalable}
\begin{frame} {Etude préalable : Tests}
	\begin{itemize}
		\item BestFirst+ CfsSubsetEval
		\pause
		\item GreedyStepwise + CfsSubsetEval
		\pause
		\item LinearForwardSelection+ CfsSubsetEval
		\pause
		\item Ranker + InfoGainAttributeEval
	\end{itemize}
\end{frame}

\begin{frame} {BestFirst + CfsSubsetEval}
	\begin{itemize}
		\item Ensemble de départ = phi , Direction = Forward
		\item Paramètres de l'algorithme : 
			\begin{itemize}
				\item[direction] = Forward
				\item[startSet] = (vide)
			\end{itemize}
	\end{itemize}
\end{frame}

\begin{frame}
\begin{description}
	\item[threshold] = \nombre{0.005}
	\item[threshold] = \nombre{0.01}
	\item[threshold] = \nombre{0.02}
	\item[threshold] = \nombre{0.03}
	\item[threshold] = \nombre{0.05}
\end{description}

\textbf{Variation du seuil}

\begin{center}
	\begin{tabular}{c c}
		\textbf{threshold} & \textbf{Nb attributs}\\
		\nombre{0.005} & 356\\
		\nombre{0.01} & 184\\
		\nombre{0.02} & 90\\
		\nombre{0.03} & 57\\
		\nombre{0.04} & 37\\
		\nombre{0.05} & 30\\
	\end{tabular}
\end{center}
\end{frame}

\subsection{Discussion}
\begin {frame}{Étude préalable -- Discussion}
Deux ensembles d'attributs distincts : 
	\begin{itemize}
		\item celui généré par les algorithmes utilisant la méthode d'évaluation \texttt{CfsSubsetEval}
		\item celui généré par les algorithmes utilisant la méthode d'évaluation \texttt{InfoGainAttributeEval}
	\end{itemize}
\end{frame}

\subsection{Nos choix}
\begin {frame} {Nos choix}
	\begin{itemize}
		\item \texttt{GreedyStepwise} car plutôt rapide.
		\pause
		\item \texttt{Ranker} en faisant varier le paramètre \og threshold \fg{}.
	\end{itemize}
\end{frame}
