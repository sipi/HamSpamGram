Dans cette partie nous avons choisi d'utiliser quatre algorithmes proposés par Weka \og~J48~\fg{}, \og~NBTree~\fg{}, \og~NaiveBayes~\fg{} et \og~KStar~\fg{}. Nous avons appliqué ces algorithmes sur l'ensemble des fichiers générés dans la partie précédente.

\subsection{Descripton des algorithmes}


\subsection{Mise en oeuvre}

Afin d'appliquer ces quatres algorithmes sur l'ensemble des fichiers générés dans la partie 3a, nous avons décidé d'écrire un petit script bash :
//todo{Thibaut} source : script.sh

\subsection{Résultats}

Après avoir exécuté les différents algorithmes, nous pouvons faire plusieurs remarques.

\subsubsection{Sélection d'attributs}
Nous pouvons observer, grâce aux différents seuils choisis pour l'algorithme de selection d'attributs \og Ranker \fg{}, que plus le nombre d'attributs selectionnés est grands plus le nombre de SMS mal classés est faible.

\subsubsection{Meilleur algorithme}
    le meilleur algo semble etre NBTree

\subsubsection{Meilleur représentation des données}
    le meilleur modèle semble être le boolean 

    l'algo KStar a une bonne précision sur les SPAM (très peu de ham classé comme SPAM)



