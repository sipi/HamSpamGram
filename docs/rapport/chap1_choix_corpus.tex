\section{Contraintes}
Dans un premier temps, nous avons dû constituer un corpus. Ce dernier devait être composé de documents \textbf{pouvant être distribués dans au moins deux classes}, chacune devant regrouper au moins 25 documents. Nous étions libre de choisir la langue du corpus.

\section{Notre choix}
Nous avons choisi de travailler sur le corpus \texttt{SMS Spam Corpus v.1}. Il comporte un ensemble de SMS en Anglais, classés en deux catégories : les \textbf{Spam} et les \textbf{Ham} (messages légitimes)\footnote{L'origine du mot Spam est \og \textbf{Sp}iced H\textbf{am} \fg{} (en Français \og Jambon épicé \fg{}), naturellement l'autre catégorie s'appelle \og Ham \fg{} (en Français \og Jambon \fg{})}.

Ce corpus a été créé à partir de données collectées depuis des données libres sur Internet :

\begin{description}
\item[JSC] : The \textbf{J}on \textbf{S}tevenson \textbf{C}orpus\footnote{JSC diponible à l'adresse suivante : \texttt{http://www.demo.inty.net/Units/SMS/corpus.htm}} est composé de 202 textes correspondant à des SMS légitimes.
\item[NSC] : The NUS SMS Corpus\footnote{NSC disponible à l'adresse suivante : \texttt{http://www.comp.nus.edu.sg/~rpnlpir/downloads/corpora/smsCorpus/}} contient environ \nombre{10000} SMS légitimes collectés pour des projets de recherche au département informatique de l'Université de Singapour. \nombre{3375} SMS ont été extrait aléatoirement de ce corpus.
\item[Grumbletext WebSite] : 425 Spam SMS extraits manuellement depuis le site web \texttt{Grumbletext}. Il s'agit d'un forum Anglais sur lequel les utilisateurs dénoncent les SPAM SMS qu'ils reçoivent.
\item[Thèse de Caroline Tagg] 450 SMS légitimes extraits de cette thèse.
\item[Corpus SMS Spam Corpus v.0.1] 1002 Ham et 322 Spam récupérés à partir de la version 0.1 de ce corpus.
\end{description}

\subsection*{Volume de données}

Au total, le \texttt{SMS Spam Corpus v.1} contient \textbf{5574} SMS, distribués dans les deux classes à raison de \textbf{747} Spam et \textbf{4827} Ham.

\subsection*{Format des données}

Les données sont fournies sous la forme d'un fichier texte contenant un SMS par ligne. Chaque ligne débute par un tag \og ham \fg{} ou \og spam \fg{} signifiant l'appartenance du SMS à leur classe respective. Ces tags sont séparés par une tabulation du texte du SMS.

Extrait du fichier :

\begin{verbatim}
ham   What you doing?how are you?
ham   Ok lar... Joking wif u oni...
spam  Sunshine Quiz! Win a super Sony DVD recorder if you canname 
the capital of Australia? Text MQUIZ to 82277. B
spam  URGENT! Your Mobile No 07808726822 was awarded a L2,000 Bonus 
Caller Prize on 02/09/03! This is our 2nd attempt to contact YOU! 
Call 0871-872-9758 BOX95QU
\end{verbatim}
